\documentclass[11pt]{book}
\usepackage[utf8]{inputenc}
\usepackage{amsmath,amsfonts,amssymb,amsthm}
\usepackage[width=16.00cm, height=24.00cm]{geometry}
\usepackage[dvipdfmx]{graphicx}
\usepackage[english]{babel}

%図の場所をなるべく指定した場所にする
\usepackage{booktabs}
\usepackage{here}

\RequirePackage[l2tabu, orthodox]{nag}
\usepackage[all, warning]{onlyamsmath}

%単位を書くときに使う
\usepackage{siunitx}

\usepackage{CJKutf8}
\usepackage{ascmac} % screen
\usepackage{ulem}
\usepackage{cases}
\usepackage{braket}
\usepackage{dsfont}
\usepackage{ascmac}
\usepackage{url}
\usepackage{hyperref} % hyper link
\usepackage{ccicons} % creative commons license icon
\usepackage{fancyhdr} % footer
%\pagestyle{fancy}
%\cfoot[\href{http://creativecommons.org/licenses/by-nc-sa/4.0/}{Creative Commons Attribution-NonCommercial-ShareAlike 4.0 International License}.]{}
\usepackage{color}


\usepackage{fancyhdr}
\setlength{\headheight}{15.2pt}
\pagestyle{fancy}
\lhead[\leftmark ]{\thepage}
\rhead[\thepage]{\leftmark}

\cfoot{\footnotesize \textcopyright 2018 goropikari - \href{http://creativecommons.org/licenses/by-nc-sa/4.0/}{Creative Commons Attribution-NonCommercial-ShareAlike 4.0 International License}}

% コマンド定義
\DeclareMathOperator{\Tr}{Tr}
\newcommand{\norm}[1]{\left\lVert#1\right\rVert} % norm ||x||
\newcommand{\kb}[1]{\ket{#1}\hspace{-1mm} \bra{#1}} % |x><x|
\newcommand{\kbt}[2]{\ket{#1}\hspace{-1mm} \bra{#2}} % |x><y|
\newcommand{\Textbf}[1]{\hspace{3mm}\\ \textbf{#1}\\}
\newtheorem{thm}{Theorem.}[section]
\newtheorem{prop}{Proposition.}[section]
\begin{document}

\Textbf{11.6}
\begin{align}
    H(Y) + H (X, Y, Z) - H(X, Y) - H(Y, Z)
        &= \sum_{x,y,z} p(x,y,z) \log \left( p(x,y)p(y,z)/p(y)p(x,y,z) \right) \\
        &\le \frac{1}{\ln{2}} \sum_{x,y,z} p(x,y,z) \left[1-p(x,y)p(y,z)/p(y)p(x,y,z) \right]\\
        &= \frac{1-1}{\ln{2}}
        = 0
\end{align}
The equality occurs if and only if $p(x,y)p(y,z)/p(y)p(x,y,z)=1$, which means a Markov chain condition of $Z \rightarrow Y \rightarrow X$,which is $p(x|y)=p(x|y,z)$
\Textbf{11.7}
\Textbf{11.8}
\Textbf{11.9}
\Textbf{11.10}
\Textbf{11.11}
\Textbf{11.12}
\Textbf{11.13}
\Textbf{11.14}
\Textbf{11.15}
\Textbf{11.16}
\Textbf{11.17}
\Textbf{11.18}
\Textbf{11.19}
\Textbf{11.20}
\Textbf{11.21}
\Textbf{11.22}
\Textbf{11.23}
\Textbf{11.24}
\Textbf{11.25}
\Textbf{11.26}








\end{document}
\\