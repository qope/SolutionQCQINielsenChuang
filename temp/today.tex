\documentclass[12pt]{book}
\usepackage[utf8]{inputenc}
\usepackage{amsmath,amsfonts,amssymb,amsthm}
\usepackage[width=16.00cm, height=24.00cm]{geometry}
\usepackage[dvipdfmx]{graphicx}
\usepackage[english]{babel}

%図の場所をなるべく指定した場所にする
\usepackage{booktabs}
\usepackage{here}

\RequirePackage[l2tabu, orthodox]{nag}
\usepackage[all, warning]{onlyamsmath}

%単位を書くときに使う
\usepackage{siunitx}

\usepackage{CJKutf8}
\usepackage{ascmac} % screen
\usepackage{ulem}
\usepackage{cases}
\usepackage{braket}
\usepackage{dsfont}
\usepackage{ascmac}
\usepackage{url}
\usepackage{hyperref} % hyper link
\usepackage{ccicons} % creative commons license icon
\usepackage{fancyhdr} % footer
%\pagestyle{fancy}
%\cfoot[\href{http://creativecommons.org/licenses/by-nc-sa/4.0/}{Creative Commons Attribution-NonCommercial-ShareAlike 4.0 International License}.]{}
\usepackage{color}


\usepackage{fancyhdr}
\setlength{\headheight}{15.2pt}
\pagestyle{fancy}
\lhead[\leftmark ]{\thepage}
\rhead[\thepage]{\leftmark}

\cfoot{\footnotesize \textcopyright 2018 goropikari - \href{http://creativecommons.org/licenses/by-nc-sa/4.0/}{Creative Commons Attribution-NonCommercial-ShareAlike 4.0 International License}}

% コマンド定義
\DeclareMathOperator{\Tr}{Tr}
\newcommand{\norm}[1]{\left\lVert#1\right\rVert} % norm ||x||
\newcommand{\kb}[1]{\ket{#1}\hspace{-1mm} \bra{#1}} % |x><x|
\newcommand{\kbt}[2]{\ket{#1}\hspace{-1mm} \bra{#2}} % |x><y|
\newcommand{\Textbf}[1]{\hspace{3mm}\\ \textbf{#1}\\}
\newtheorem{thm}{Theorem.}[section]
\newtheorem{prop}{Proposition.}[section]
\begin{document}



\Textbf{12.1}
\Textbf{12.2}
\Textbf{12.3}
Equality of (12.14) will happen when $\rho_x$ has orthogonal support. It is obvious that n qubits have at most n orthogonal $\rho_x$s, and from (12.6), 
\begin{align}
H(X:Y) \le \chi(\rho) \le H(X) \le n
\end{align}
So, n qubits can be used to at most n bits of classical information.
\Textbf{12.6}
\Textbf{12.7}

\Textbf{12.8}
\Textbf{12.9}
\Textbf{12.10}
\Textbf{12.12}
\Textbf{12.12}
\Textbf{12.13}
\Textbf{12.14}
\Textbf{12.15}
\Textbf{12.16}
\Textbf{12.17}
\Textbf{12.18}
\Textbf{12.19}
\Textbf{12.20}
\Textbf{12.21}
\Textbf{12.22}
\Textbf{12.23}
\Textbf{12.24}
\Textbf{12.25}
\Textbf{12.26}
\Textbf{12.31}
Eve makes her qubits entangled with $\ket{\beta_{00}}$, and gets $\rho^E$.
\begin{align}
\ket{ABE} = U\ket{\beta_{00}^{\otimes n}}\ket{0}_E\\
\rho^E = tr_{AB} (\ket{ABE} \bra{ABE})
\end{align}
Note that Eve's mutual information with Alice and Bob measurements does not depend on whether Eve measures $\rho^E$ before Alice and Bob's measurement or after.
So we can assume that Eve measures $\rho^E$ after Alice and Bob's measurement.
Alice and Bob measure their Bell state, getting binary string $\vec{k}$ as an outcome.
Let $\rho^E_k$ and $p_k$ are the corresponding Eve's states and probabilities.
Note,
\begin{align}
\rho_E = \sum_k p_k \rho^E_k.
\end{align}
Let $K$ is a variable of $\vec{k}$ and $e$ is an outcom of a measurement of $\rho^E$, and $E$ is its variable.  From Holevo bound, 
\begin{align}
H(K:E) \le S(\rho^E) - \sum_k p_k \rho^E_k \le S(\rho^E) = S(\rho).
\end{align}


\end{document}
\\