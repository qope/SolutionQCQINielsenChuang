%
% generate PDF file
% pdflatex solnQCQI.tex
%
\documentclass[11pt]{book}
\usepackage[utf8]{inputenc}
\usepackage{amsmath,amsfonts,amssymb,amsthm}
\usepackage[width=16.00cm, height=24.00cm]{geometry}
\usepackage[dvipdfmx]{graphicx}
\usepackage[english]{babel}

%図の場所をなるべく指定した場所にする
\usepackage{booktabs}
\usepackage{here}

\RequirePackage[l2tabu, orthodox]{nag}
\usepackage[all, warning]{onlyamsmath}

%単位を書くときに使う
\usepackage{siunitx}

\usepackage{CJKutf8}
\usepackage{ascmac} % screen
\usepackage{ulem}
\usepackage{cases}
\usepackage{braket}
\usepackage{dsfont}
\usepackage{ascmac}
\usepackage{url}
\usepackage{hyperref} % hyper link
\usepackage{ccicons} % creative commons license icon
\usepackage{fancyhdr} % footer
%\pagestyle{fancy}
%\cfoot[\href{http://creativecommons.org/licenses/by-nc-sa/4.0/}{Creative Commons Attribution-NonCommercial-ShareAlike 4.0 International License}.]{}
\usepackage{color}


\usepackage{fancyhdr}
\setlength{\headheight}{15.2pt}
\pagestyle{fancy}
\lhead[\leftmark ]{\thepage}
\rhead[\thepage]{\leftmark}

\cfoot{\footnotesize \textcopyright 2018 goropikari - \href{http://creativecommons.org/licenses/by-nc-sa/4.0/}{Creative Commons Attribution-NonCommercial-ShareAlike 4.0 International License}}

% コマンド定義
\DeclareMathOperator{\Tr}{Tr}
\newcommand{\norm}[1]{\left\lVert#1\right\rVert} % norm ||x||
\newcommand{\kb}[1]{\ket{#1}\hspace{-1mm} \bra{#1}} % |x><x|
\newcommand{\kbt}[2]{\ket{#1}\hspace{-1mm} \bra{#2}} % |x><y|
\newcommand{\Textbf}[1]{\hspace{3mm}\\ \textbf{#1}\\}
\newtheorem{thm}{Theorem.}[section]
\newtheorem{prop}{Proposition.}[section]


\title{Solution for "Quantum Computation and Quantum Information: 10th Anniversary Edition" by Nielsen and Chuang}
\author{goropikari}
\date{\today}

\begin{document}
\maketitle
\thispagestyle{empty}
\setcounter{page}{0} % 表紙のページを0ページにする

\section*{Copylight Notice:}
\ccbyncsa\\
This work is licensed under a \href{http://creativecommons.org/licenses/by-nc-sa/4.0/}{Creative Commons Attribution-NonCommercial-ShareAlike 4.0 International License}.


\section*{Repository}
The latest version and source \LaTeX code are located in\\ \url{https://github.com/goropikari/SolutionForQuantumComputationAndQuantumInformation}.

\section*{For readers}
This is an unofficial solution manual for "\href{http://www.cambridge.org/jp/academic/subjects/physics/quantum-physics-quantum-information-and-quantum-computation/quantum-computation-and-quantum-information-10th-anniversary-edition?format=HB&isbn=9781107002173#BBFv83H3ofgcgG3A.97}{Quantum Computation and Quantum Information: 10th Anniversary Edition}" (ISBN-13: 978-1107002173) by Michael A. Nielsen and Isaac L. Chuang.



I have studied quantum information theory as a hobby.
And I'm not a researcher.
So there is no guarantee that these solutions are correct.
Especially because I'm not good at mathematics, proofs are often wrong.
Don't trust me. Verify yourself!

If you find some mistake or have some comments, please feel free to open an issue or a PR.
\begin{flushright}
    \href{https://github.com/goropikari}{goropikari}
\end{flushright}

\tableofcontents
\newpage

%%%%%%%%%%%%%%%%%%%%%%%%%%%%%%%%%%%%%%%%%%%%%%%%%%%%%%%%%%%%%%%%%%%%%%%%%%%%%
\frontmatter
\include{chapter/errata}

\mainmatter
\input{chapter/chapter2}
\setcounter{chapter}{7}
\input{chapter/chapter8}
\input{chapter/chapter9}
\setcounter{chapter}{10}
\chapter{Entropy and information}
\Textbf{11.1}
Fair coin:

\begin{align}
    H({1/2, 1/2}) = \left( - \frac{1}{2} \log \frac{1}{2} \right) \times 2 = 1
\end{align}


Fair die:
\begin{align}
    H(p) = \left( - \frac{1}{6} \log \frac{1}{6} \right) \times 6 = \log 6.
\end{align}


The entropy decreases if the coin or die is unfair.



\Textbf{11.2}

From assumption $I(pq) = I(p) + I(q)$.

\begin{align}
    \frac{\partial I(pq)}{\partial p} &= \frac{\partial I(p)}{\partial p} + 0 = \frac{\partial I(p)}{\partial p}\\
    \frac{\partial I(pq)}{\partial q} &= 0 + \frac{\partial I(q)}{\partial q} = \frac{\partial I(q)}{\partial q}
\end{align}


\begin{align}
    \frac{\partial I(pq)}{\partial p}
        &=  \frac{\partial I(pq)}{\partial (pq)} \frac{\partial (pq)}{\partial p}
        = q \frac{\partial I(pq)}{\partial(pq)}
    \Rightarrow \frac{\partial I(pq)}{\partial(pq)} = \frac{1}{q} \frac{\partial I(p)}{\partial p}\\
%
    \frac{\partial I(pq)}{\partial q}
        &=  \frac{\partial I(pq)}{\partial (pq)} \frac{\partial (pq)}{\partial q}
        = p \frac{\partial I(pq)}{\partial(pq)}
    \Rightarrow \frac{\partial I(pq)}{\partial(pq)} = \frac{1}{p} \frac{\partial I(q)}{\partial q}
\end{align}

Thus
\begin{align}
    \frac{1}{q} \frac{\partial I(p)}{\partial p} &= \frac{1}{p} \frac{\partial I(q)}{\partial q}\\
    \therefore~ p \frac{d I(p)}{d p} &= q \frac{d I(q)}{d q} ~~~\text{ for all } p,q \in [0,1].\\
\end{align}

Then $p (d I(p) / d p)$ is constant.

If $p (d I(p) / d p) = k$, $k \in \mathds{R}$.
Then $I(p) = k \ln p = k' \log p$ where $k' = k / \log e$.



\Textbf{11.3}
$H_{\text{bin}}(p) = - p\log p - (1-p) \log (1-p)$.

\begin{align}
    \frac{d H_{\text{bin}}(p)}{d p}
        &= \frac{1}{\ln 2} \left( - \log p - 1 + \log (1-p) + 1 \right)\\
        &= \frac{1}{\ln 2} \ln \frac{1-p}{p} = 0\\
    \Rightarrow \frac{1-p}{p} = 1\\
    \Rightarrow p = 1/2.
\end{align}


\Textbf{11.4}
\Textbf{11.5}

\begin{align}
    H\left( p(x,y) || p(x)p(y) \right)
        &= \sum_{x,y} p(x,y) \log \frac{p(x) p(y)}{p(x,y)}\\
        &= - H(p(x,y)) - \sum_{x,y} p(x,y) \log \left[ p(x)p(y) \right]\\
        &= - H(p(x,y)) - \sum_{x,y} p(x,y) \left[ \log p(x) + \log p(y) \right]\\
        &= - H(p(x,y)) - \sum_{x,y} p(x,y) \log p(x) - \sum_{x,y} p(x,y) \log p(y)\\
        &= - H(p(x,y)) - \sum_{x} p(x) \log p(x) - \sum_{y} p(y) \log p(y)\\
        &= - H(p(x,y)) + H(p(x)) + H(p(y))\\
        &= - H(X,Y) + H(X) + H(Y).
\end{align}

From the non-negativity of the relative entropy,
\begin{align}
    H(X) +  H(Y) - H(X,Y) \geq 0\\
    \therefore H(X) + H(Y) \geq H(X,Y).
\end{align}



\Textbf{11.6}
\begin{align}
    H(Y) + H (X, Y, Z) - H(X, Y) - H(Y, Z)  
        &= \sum_{x,y,z} p(x,y,z) \log \left( p(x,y)p(y,z)/p(y)p(x,y,z) \right) \\
        &\geq \frac{1}{\ln{2}} \sum_{x,y,z} p(x,y,z) \left[1-p(x,y)p(y,z)/p(y)p(x,y,z) \right] \\
        = \frac{1-1}{\ln{2}}
        = 0
\end{align}
The equality occurs if and only if $p(x,y)p(y,z)/p(y)p(x,y,z)=1$, which means a Markov chain condition of $Z \rightarrow Y \rightarrow X$; $p(x|y)=p(x|y,z)$ 
\Textbf{11.7}
\Textbf{11.8}
\Textbf{11.9}
\Textbf{11.10}
\Textbf{11.11}
\Textbf{11.12}
\Textbf{11.13}
\Textbf{11.14}
\Textbf{11.15}
\Textbf{11.16}
\Textbf{11.17}
\Textbf{11.18}
\Textbf{11.19}
\Textbf{11.20}
\Textbf{11.21}
\Textbf{11.22}
\Textbf{11.23}
\Textbf{11.24}
\Textbf{11.25}
\Textbf{11.26}

\Textbf{Problem 11.1}
\Textbf{Problem 11.2}
\Textbf{Problem 11.3}
\Textbf{Problem 11.4}
\Textbf{Problem 11.5}


\Textbf{12.31}
Eve makes her qubits entangled with $\ket{\beta_{00}}$, and gets $\rho^E$.
\begin{align}
\ket{ABE} = U\ket{\beta_{00}^{\otimes n}}\ket{0}_E\\
\rho^E = tr_{AB} (\ket{ABE} \bra{ABE})
\end{align}
Note that Eve's mutual information with Alice and Bob measurements does not depend on whether Eve measures $\rho^E$ before Alice and Bob's measurement or after.
So we can assume that Eve measures $\rho^E$ after Alice and Bob's measurement.
Alice and Bob measure their Bell state, getting binary string $\vec{k}$ as an outcome.
Let $\rho^E_k$ and $p_k$ are the corresponding Eve's states and probabilities.
Note,
\begin{align}
\rho_E = \sum_k p_k \rho^E_k.
\end{align}
Let $K$ is a variable of $\vec{k}$ and $e$ is an outcom of a measurement of $\rho^E$, and $E$ is its variable.  From Holevo bound, 
\begin{align}
H(K:E) \le S(\rho^E) - \sum_k p_k \rho^E_k \le S(\rho^E) = S(\rho).
\end{align}


%%%%%%%%%%%%%%%%%%%%%%%%%%%%%%%%%%%%%%%%%%%%%%%%%%%%%%%%%%%%%%%%%%%%%%%%%%%%%

%参考文献
%\bibliographystyle{jplain}
%\bibliography{ref} % ref.bib を読み込み
\end{document}
